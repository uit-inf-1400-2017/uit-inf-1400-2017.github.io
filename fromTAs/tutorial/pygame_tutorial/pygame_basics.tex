\documentclass[12pt, norsk, a4paper]{article}
\usepackage[utf8]{inputenc}
\usepackage[pdftex]{color, graphicx}
\usepackage{graphicx, babel}
\usepackage{url}

\renewcommand{\labelenumi}{(\alph{enumi})}

% LaTeX er normalt streng når det gjelder linjebrytingen. Vi vil være litt
% mildere, særlig fordi norsk har så mange lange, sammensatte ord
\tolerance = 5000
\hbadness = \tolerance
\pretolerance = 2000

% For å få avsnitt uten innrykk:
\setlength{\parindent}{0pt}
\setlength{\parskip}{2ex plus 0.2ex minus 0.2ex}
\setlength\fboxsep{10pt}
\setlength\fboxrule{0.5pt}

\title{INF-1400 Introduksjon til PyGame}
%\author{Øyvind Holmstad, Martin Ernstsen, Erik Bræck Leer, Sigurd Eilertsen}
\date{21. januar, 2014}

\begin{document}
\maketitle

\section*{Informasjon}
Pygame er en ``wrapper'' rundt SDL-biblioteket som ble brukt i INF-1100. Det
betyr at dere vil kjenne dere igjen i måten man bruker pygame på. Dokumentasjon
til pygame finnes på \url{www.pygame.org} under \emph{Documentation}. Øverst på
siden finner dere en oversikt over alle klassene. Klikk på aktuell klasse for å
se hvilke metoder denne har. Dere finner også brukereksempler på hvordan
metodene brukes.

Her er et minimalistisk pygame-eksempel for å få dere i gang:
\begin{verbatim}
import pygame

class Canvas(object):
    # A drawing canvas.
    def __init__(self):
        # Set up the pygame window.
        pygame.init()
        pygame.display.set_mode((400, 300))
        pygame.display.set_caption("Pygame Demo")
        self.screen = pygame.display.get_surface()

    def handle_events(self):
        # Handle all events.
        for event in pygame.event.get():
            if event.type == pygame.QUIT:
                exit()

    def update(self):
        # Change position of objects. Nothing here yet.
        pass

    def draw_scene(self):
        # Draw a circle in the window.
        pygame.draw.circle(self.screen, pygame.Color("yellow"), (50, 50), 30, 2)
        pygame.display.flip()

    def run(self):
        # The main loop.
        while 1:
            self.handle_events()
            self.update()
            self.draw_scene()

if __name__ == '__main__':
    cv = Canvas()
    cv.run()
\end{verbatim}
I hovedløkka tegnes en gul sirkel med sentrum i $(50, 50)$, radius 30 pixler og
strekbredde 2 pixler.

\section{Tegning}
\begin{enumerate}
\item Skriv et program der du tegner sirkler av forskjellig størrelse. Prøv å
forandre alle parametre i draw-funksjonen.
\item Bruk pygame til å tegne OL-ringene.
\end{enumerate}

\section{Animasjon}
I denne oppgaven tar du utgangspunkt i eksempelprogrammet over.
\begin{enumerate}
\item Legg til to variabler i initmetoden, self.pos\_x og self.pos\_y og gi begge
startverdi 50.
\item Endre posisjons tuppelen i linjen som tegner sirkelen, til å 
bruke de to nye variablene.
\item Legg til kode i updatemetoden som endrer èn eller begge posisjonsvariablene.
\item Endre \texttt{draw\_scene} slik at den gamle sirkelen blir visket ut
(bruk fillmetoden til self.screen, som er en instans av pygame.Surface-klassen).
\end{enumerate}

\section{Events}
Fortsetter med koden fra forrige oppgave.
\begin{enumerate}
\item Legg til 4 nye variabler i \texttt{init}: left\_pressed, right\_pressed, up\_pressed, down\_pressed.
Initialiser disse til False.
\item Legg til event-sjekker for om piltastene er trykket ned eller sluppet opp
i \texttt{handle\_events}. Test etter pygame.KEYUP og pygame.KEYDOWN som event.type.
Deretter hent ut verdiene i event.key (piltastene er pygame.K\_UP, pygame.K\_DOWN, pygame.K\_RIGHT og pygame.K\_LEFT).
Se \url{http://pygame.org/docs/ref/event.html} for mer dokumentasjon.

\item I piltast testene, gjør endring til \_pressed variablene ut ifra hvilken piltast som er trykket ned eller sluppet opp.

\item I \texttt{update}, bruk \_pressed variablene til å vite hvilken piltaster er trykket ned, og gjør tilsvarende endringer til pos\_x og pos\_y.
\end{enumerate}

\end{document}
